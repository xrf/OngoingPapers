 
\documentclass[compress]{beamer}


% Try the class options [notes], [notes=only], [trans], [handout],
% [red], [compress], [draft], [class=article] and see what happens!

% For a green structure color use:
%\colorlet{structure}{green!50!black}

\mode<article> % only for the article version
{
  \usepackage{beamerbasearticle}
  \usepackage{fullpage}
  \usepackage{hyperref}
}

\beamertemplateshadingbackground{red!10}{blue!10}

%\beamertemplateshadingbackground{red!10}{blue!10}
\beamertemplatetransparentcovereddynamic
%\usetheme{Hannover}

\setbeamertemplate{footline}[page number]


%\usepackage{beamerthemeshadow}

%\usepackage{beamerthemeshadow}
\usepackage{ucs}


\usepackage{pgf,pgfarrows,pgfnodes,pgfautomata,pgfheaps,pgfshade}
\usepackage{graphicx}
\usepackage{simplewick}
\usepackage{amsmath,amssymb}
\usepackage[latin1]{inputenc}
\usepackage{colortbl}
\usepackage[english]{babel}
\usepackage{listings}
\usepackage{shadow}
\lstset{language=c++}
\lstset{alsolanguage=[90]Fortran}
\lstset{basicstyle=\small}
%\lstset{backgroundcolor=\color{white}}
%\lstset{frame=single}
\lstset{stringstyle=\ttfamily}
%\lstset{keywordstyle=\color{red}\bfseries}
%\lstset{commentstyle=\itshape\color{blue}}
\lstset{showspaces=false}
\lstset{showstringspaces=false}
\lstset{showtabs=false}
\lstset{breaklines}
\usepackage{times}

% Use some nice templates
\beamertemplatetransparentcovereddynamic

% own commands
\newcommand*{\cre}[1]{a^{\dagger}_{#1}}
\newcommand*{\an}[1]{a_{#1}}
\newcommand*{\crequasi}[1]{b^{\dagger}_{#1}}
\newcommand*{\anquasi}[1]{b_{#1}}
\newcommand*{\for}[3]{\langle#1|#2|#3\rangle}
\newcommand{\be}{\begin{equation}}
\newcommand{\ee}{\end{equation}}
\newcommand*{\kpr}[1]{\left\{#1\right\}}
\newcommand*{\ket}[1]{|#1\rangle}
\newcommand*{\bra}[1]{\langle#1|}
%\newcommand{\bra}[1]{\left\langle #1 \right|}
%\newcommand{\ket}[1]{\left| # \right\rangle}
\newcommand{\braket}[2]{\left\langle #1 \right| #2 \right\rangle}
\newcommand{\OP}[1]{{\bf\widehat{#1}}}
\newcommand{\matr}[1]{{\bf \cal{#1}}}
\newcommand{\beN}{\begin{equation*}}
\newcommand{\bea}{\begin{eqnarray}}
\newcommand{\beaN}{\begin{eqnarray*}}
\newcommand{\eeN}{\end{equation*}}
\newcommand{\eea}{\end{eqnarray}}
\newcommand{\eeaN}{\end{eqnarray*}}
\newcommand{\bdm}{\begin{displaymath}}
\newcommand{\edm}{\end{displaymath}}
\newcommand{\bsubeqs}{\begin{subequations}}
\newcommand*{\fpr}[1]{\left[#1\right]}
\newcommand{\esubeqs}{\end{subequations}}
\newcommand*{\pr}[1]{\left(#1\right)}
\newcommand{\element}[3]
        {\bra{#1}#2\ket{#3}}

\newcommand{\md}{\mathrm{d}}
\newcommand{\e}[1]{\times 10^{#1}}
\renewcommand{\vec}[1]{\mathbf{#1}}
\newcommand{\gvec}[1]{\boldsymbol{#1}}
\newcommand{\dr}{\, \mathrm d^3 \vec r}
\newcommand{\dk}{\, \mathrm d^3 \vec k}
\def\psii{\psi_{i}}
\def\psij{\psi_{j}}
\def\psiij{\psi_{ij}}
\def\psisq{\psi^2}
\def\psisqex{\langle \psi^2 \rangle}
\def\psiR{\psi({\bf R})}
\def\psiRk{\psi({\bf R}_k)}
\def\psiiRk{\psi_{i}(\Rveck)}
\def\psijRk{\psi_{j}(\Rveck)}
\def\psiijRk{\psi_{ij}(\Rveck)}
\def\ranglep{\rangle_{\psisq}}
\def\Hpsibypsi{{H \psi \over \psi}}
\def\Hpsiibypsi{{H \psii \over \psi}}
\def\HmEpsibypsi{{(H-E) \psi \over \psi}}
\def\HmEpsiibypsi{{(H-E) \psii \over \psi}}
\def\HmEpsijbypsi{{(H-E) \psij \over \psi}}
\def\psiibypsi{{\psii \over \psi}}
\def\psijbypsi{{\psij \over \psi}}
\def\psiijbypsi{{\psiij \over \psi}}
\def\psiibypsiRk{{\psii(\Rveck) \over \psi(\Rveck)}}
\def\psijbypsiRk{{\psij(\Rveck) \over \psi(\Rveck)}}
\def\psiijbypsiRk{{\psiij(\Rveck) \over \psi(\Rveck)}}
\def\EL{E_{\rm L}}
\def\ELi{E_{{\rm L},i}}
\def\ELj{E_{{\rm L},j}}
\def\ELRk{E_{\rm L}(\Rveck)}
\def\ELiRk{E_{{\rm L},i}(\Rveck)}
\def\ELjRk{E_{{\rm L},j}(\Rveck)}
\def\Ebar{\bar{E}}
\def\Ei{\Ebar_{i}}
\def\Ej{\Ebar_{j}}
\def\Ebar{\bar{E}}
\def\Rvec{{\bf R}}
\def\Rveck{{\bf R}_k}
\def\Rvecl{{\bf R}_l}
\def\NMC{N_{\rm MC}}
\def\sumMC{\sum_{k=1}^{\NMC}}
\def\MC{Monte Carlo}
\def\adiag{a_{\rm diag}}
\def\tcorr{T_{\rm corr}}
\def\intR{{\int {\rm d}^{3N}\!\!R\;}}

\def\ul{\underline}
\def\beq{\begin{eqnarray}}
\def\eeq{\end{eqnarray}}

\newcommand{\eqbrace}[4]{\left\{
\begin{array}{ll}
#1 & #2 \\[0.5cm]
#3 & #4
\end{array}\right.}
\newcommand{\eqbraced}[4]{\left\{
\begin{array}{ll}
#1 & #2 \\[0.5cm]
#3 & #4
\end{array}\right\}}
\newcommand{\eqbracetriple}[6]{\left\{
\begin{array}{ll}
#1 & #2 \\
#3 & #4 \\
#5 & #6
\end{array}\right.}
\newcommand{\eqbracedtriple}[6]{\left\{
\begin{array}{ll}
#1 & #2 \\
#3 & #4 \\
#5 & #6
\end{array}\right\}}

\newcommand{\mybox}[3]{\mbox{\makebox[#1][#2]{$#3$}}}
\newcommand{\myframedbox}[3]{\mbox{\framebox[#1][#2]{$#3$}}}

%% Infinitesimal (and double infinitesimal), useful at end of integrals
%\newcommand{\ud}[1]{\mathrm d#1}
\newcommand{\ud}[1]{d#1}
\newcommand{\udd}[1]{d^2\!#1}

%% Operators, algebraic matrices, algebraic vectors

%% Operator (hat, bold or bold symbol, whichever you like best):
\newcommand{\op}[1]{\widehat{#1}}
%\newcommand{\op}[1]{\mathbf{#1}}
%\newcommand{\op}[1]{\boldsymbol{#1}}

%% Vector:
\renewcommand{\vec}[1]{\boldsymbol{#1}}

%% Matrix symbol:
%\newcommand{\matr}[1]{\boldsymbol{#1}}
%\newcommand{\bb}[1]{\mathbb{#1}}

%% Determinant symbol:
\renewcommand{\det}[1]{|#1|}

%% Means (expectation values) of varius sizes
\newcommand{\mean}[1]{\langle #1 \rangle}
\newcommand{\meanb}[1]{\big\langle #1 \big\rangle}
\newcommand{\meanbb}[1]{\Big\langle #1 \Big\rangle}
\newcommand{\meanbbb}[1]{\bigg\langle #1 \bigg\rangle}
\newcommand{\meanbbbb}[1]{\Bigg\langle #1 \Bigg\rangle}

%% Shorthands for text set in roman font
\newcommand{\prob}[0]{\mathrm{Prob}} %probability
\newcommand{\cov}[0]{\mathrm{Cov}}   %covariance
\newcommand{\var}[0]{\mathrm{Var}}   %variancd

%% Big-O (typically for specifying the speed scaling of an algorithm)
\newcommand{\bigO}{\mathcal{O}}

%% Real value of a complex number
\newcommand{\real}[1]{\mathrm{Re}\!\left\{#1\right\}}

%% Quantum mechanical state vectors and matrix elements (of different sizes)
%\newcommand{\bra}[1]{\langle #1 |}
\newcommand{\brab}[1]{\big\langle #1 \big|}
\newcommand{\brabb}[1]{\Big\langle #1 \Big|}
\newcommand{\brabbb}[1]{\bigg\langle #1 \bigg|}
\newcommand{\brabbbb}[1]{\Bigg\langle #1 \Bigg|}
%\newcommand{\ket}[1]{| #1 \rangle}
\newcommand{\ketb}[1]{\big| #1 \big\rangle}
\newcommand{\ketbb}[1]{\Big| #1 \Big\rangle}
\newcommand{\ketbbb}[1]{\bigg| #1 \bigg\rangle}
\newcommand{\ketbbbb}[1]{\Bigg| #1 \Bigg\rangle}
%\newcommand{\overlap}[2]{\langle #1 | #2 \rangle}
\newcommand{\overlapb}[2]{\big\langle #1 \big| #2 \big\rangle}
\newcommand{\overlapbb}[2]{\Big\langle #1 \Big| #2 \Big\rangle}
\newcommand{\overlapbbb}[2]{\bigg\langle #1 \bigg| #2 \bigg\rangle}
\newcommand{\overlapbbbb}[2]{\Bigg\langle #1 \Bigg| #2 \Bigg\rangle}
\newcommand{\bracket}[3]{\langle #1 | #2 | #3 \rangle}
\newcommand{\bracketb}[3]{\big\langle #1 \big| #2 \big| #3 \big\rangle}
\newcommand{\bracketbb}[3]{\Big\langle #1 \Big| #2 \Big| #3 \Big\rangle}
\newcommand{\bracketbbb}[3]{\bigg\langle #1 \bigg| #2 \bigg| #3 \bigg\rangle}
\newcommand{\bracketbbbb}[3]{\Bigg\langle #1 \Bigg| #2 \Bigg| #3 \Bigg\rangle}
\newcommand{\projection}[2]
{| #1 \rangle \langle  #2 |}
\newcommand{\projectionb}[2]
{\big| #1 \big\rangle \big\langle #2 \big|}
\newcommand{\projectionbb}[2]
{ \Big| #1 \Big\rangle \Big\langle #2 \Big|}
\newcommand{\projectionbbb}[2]
{ \bigg| #1 \bigg\rangle \bigg\langle #2 \bigg|}
\newcommand{\projectionbbbb}[2]
{ \Bigg| #1 \Bigg\rangle \Bigg\langle #2 \Bigg|}


%% If you run out of greek symbols, here's another one you haven't
%% thought of:
\newcommand{\Feta}{\hspace{0.6ex}\begin{turn}{180}
        {\raisebox{-\height}{\parbox[c]{1mm}{F}}}\end{turn}}
\newcommand{\feta}{\hspace{-1.6ex}\begin{turn}{180}
        {\raisebox{-\height}{\parbox[b]{4mm}{f}}}\end{turn}}




\title[PHY981]{Homogeneous Electron gas}
\author[Nuclear Structure]{%
  Morten Hjorth-Jensen}
\institute[ORNL, University of Oslo and MSU]{
Department of Physics and Center of Mathematics for Applications\\
  University of Oslo, N-0316 Oslo, Norway and\\
  National Superconducting Cyclotron Laboratory, Michigan State University, East Lansing, MI 48824, USA }

\date[UiO]{June 28, 2013 }
\subject{Homogeneous Electron Gas}
  

\begin{document}
\include{commands}

\frame{\titlepage}



\frame[containsverbatim]
{
  \frametitle{The electron gas}
\begin{small}
{\scriptsize
The electron gas is perhaps the only realistic model of a 
system of many interacting particles that allows for a solution
of the Hartree-Fock equations on a closed form. Furthermore, to first order in the interaction, one can also
compute on a closed form the total energy and several other properties of a many-particle systems. 
The model gives a very good approximation to the properties of valence electrons in metals.
The assumptions are
\begin{itemize}
\item System of electrons that is not influenced by external forces except by an attraction provided by a uniform background of ions. These ions give rise to a uniform background charge. The ions are stationary.
\item The system as a whole is neutral.
\item We assume we have $N_e$ electrons in a cubic box of length $L$ and volume $\Omega=L^3$. This volume contains also a
uniform distribution of positive charge with density $N_ee/\Omega$. 
\end{itemize}
}
 \end{small}
 }


\frame[containsverbatim]
{
  \frametitle{The electron gas}
\begin{small}
{\scriptsize
This is a homogeneous system and the one-particle wave functions are given by plane wave functions normalized to a volume $\Omega$ 
for a box with length $L$ (the limit $L\rightarrow \infty$ is to be taken after we have computed various expectation values)
\[
\psi_{{\bf k}\sigma}({\bf r})= \frac{1}{\sqrt{\Omega}}\exp{(i{\bf kr})}\xi_{\sigma}
\]
where ${\bf k}$ is the wave number and  $\xi_{\sigma}$ is a spin function for either spin up or down
\[ 
\xi_{\sigma=+1/2}=\left(\begin{array}{c} 1 \\ 0 \end{array}\right) \hspace{0.5cm}
\xi_{\sigma=-1/2}=\left(\begin{array}{c} 0 \\ 1 \end{array}\right).\]

We assume that we have periodic boundary conditions which limit the allowed wave numbers to
\[
k_i=\frac{2\pi n_i}{L}\hspace{0.5cm} i=x,y,z \hspace{0.5cm} n_i=0,\pm 1,\pm 2, \dots
\]
We assume first that the electrons interact via a central, symmetric and translationally invariant
interaction  $V(r_{12})$ with
$r_{12}=|{\bf r}_1-{\bf r}_2|$.  The interaction is spin independent.

The total Hamiltonian consists then of kinetic and potential energy
\[
\hat{H} = \hat{T}+\hat{V}.
\]
The operator for the kinetic energy can be written as
\[
\hat{T}=\sum_{{\bf k}\sigma}\frac{\hbar^2k^2}{2m}a_{{\bf k}\sigma}^{\dagger}a_{{\bf k}\sigma}.
\]

}
 \end{small}
 }


\frame[containsverbatim]
{
  \frametitle{The electron gas}
\begin{small}
{\scriptsize
The Hamilton operator is given by
\[
\hat{H}=\hat{H}_{el}+\hat{H}_{b}+\hat{H}_{el-b},
\]
with the electronic part
\[
\hat{H}_{el}=\sum_{i=1}^N\frac{p_i^2}{2m}+\frac{e^2}{2}\sum_{i\ne j}\frac{e^{-\mu |{\bf r}_i-{\bf r}_j|}}{|{\bf r}_i-{\bf r}_j|},
\]
where we have introduced an explicit convergence factor
(the limit $\mu\rightarrow 0$ is performed after having calculated the various integrals).
Correspondingly, we have
\[
\hat{H}_{b}=\frac{e^2}{2}\int\int d{\bf r}d{\bf r}'\frac{n({\bf r})n({\bf r}')e^{-\mu |{\bf r}-{\bf r}'|}}{|{\bf r}-{\bf r}'|},
\]
which is the energy contribution from the positive background charge with density
$n({\bf r})=N/\Omega$. Finally,
\[
\hat{H}_{el-b}=-\frac{e^2}{2}\sum_{i=1}^N\int d{\bf r}\frac{n({\bf r})e^{-\mu |{\bf r}-{\bf x}_i|}}{|{\bf r}-{\bf x}_i|},
\]
is the interaction between the electrons and the positive background.
}
 \end{small}
 }



\frame[containsverbatim]
{
  \frametitle{The electron gas}
\begin{small}
{\scriptsize
One can show that the Hartree-Fock energy can be written as 
\[
\varepsilon_{k}^{HF}=\frac{\hbar^{2}k^{2}}{2m_e}-\frac{e^{2}}
{\Omega^{2}}\sum_{k'\leq
k_{F}}\int d\vec{r}e^{i(\vec{k'}-\vec{k})\vec{r}}\int
d\vec{r'}\frac{e^{i(\vec{k}-\vec{k'})\vec{r'}}}
{\vert\vec{r}-\vec{r'}\vert}
\]
resulting in
\[
\varepsilon_{k}^{HF}=\frac{\hbar^{2}k^{2}}{2m_e}-\frac{e^{2}
k_{F}}{2\pi}
\left[
2+\frac{k_{F}^{2}-k^{2}}{kk_{F}}ln\left\vert\frac{k+k_{F}}
{k-k_{F}}\right\vert
\right]
\]
}
 \end{small}
 }

\frame[containsverbatim]
{
  \frametitle{The electron gas}
\begin{small}
{\scriptsize
The energy can be rewritten in terms of the density
\[
n= \frac{k_F^3}{3\pi^2}=\frac{3}{4\pi r_s^3},
\]
where $n=N_e/\Omega$, $N_e$ being the number of electrons, and $r_s$ is the radius of a sphere which represents the volum per conducting electron.  
It is convenient to use the Bohr radius $a_0=\hbar^2/e^2m_e$.
For most metals we have a relation $r_s/a_0\sim 2-6$.
}
 \end{small}
 }




\frame[containsverbatim]
{
  \frametitle{The electron gas, total energy}
\begin{small}
{\scriptsize
We have
\[
\hat{H}_{b}=\frac{e^2}{2}\frac{N_e^2}{\Omega}\frac{4\pi}{\mu^2},
\]
and
\[
\hat{H}_{el-b}=-e^2\frac{N_e^2}{\Omega}\frac{4\pi}{\mu^2}.
\]
The final Hamiltonian can be written as 
\[
H=H_{0}+H_{I},
\]
with
\[
H_{0}={\displaystyle\sum_{{\bf k}\sigma}}
\frac{\hbar^{2}k^{2}}{2m_e}a_{{\bf k}\sigma}^{\dagger}
a_{{\bf k}\sigma},
\]
and
\[
H_{I}=\frac{e^{2}}{2\Omega}{\displaystyle\sum_{\sigma_{1}
\sigma_{2}}}{\displaystyle
\sum_{{\bf q}\neq 0,{\bf k},{\bf p}}}\frac{4\pi}{q^{2}}
a_{{\bf k}+{\bf q},\sigma_{1}}^{\dagger}
a_{{\bf p}-{\bf q},\sigma_{2}}^{\dagger}
a_{{\bf p}\sigma_{2}}a_{{\bf k}\sigma_{1}}.
\] 

}
 \end{small}
 }



\frame[containsverbatim]
{
  \frametitle{The electron gas, total energy}
\begin{small}
{\scriptsize
We can
calculate
$E_0/N_e=\bra{\Phi_{0}}H\ket{\Phi_{0}}/N_e$ for
for this system to first order in the interaction.  
Using
\[
\rho= \frac{k_F^3}{3\pi^2}=\frac{3}{4\pi r_0^3},
\]
with $\rho=N_e/\Omega$, $r_0$
being the radius of a sphere representing the volume an electron occupies 
and the Bohr radius $a_0=\hbar^2/e^2m$, 
that the energy per electron can be written as 
\[
E_0/N_e=\frac{e^2}{2a_0}\left[\frac{2.21}{r_s^2}-\frac{0.916}{r_s}\right].
\]
Here we have defined
$r_s=r_0/a_0$ to be a dimensionless quantity.

}
 \end{small}
 }


\frame[containsverbatim]
{
  \frametitle{The electron gas, total energy}
\begin{small}
{\scriptsize
We can calculate the following part of the Hamiltonian using a convergence factor
\[ \hat H_b = \frac{e^2}{2} \iint \frac{n(\vec r) n(\vec r')e^{-\mu|\vec r - \vec r'|}}{|\vec r - \vec r'|} \dr \dr' , \]
%
where $n(\vec r) = N_e/\Omega$, the density of the positive backgroun charge. We define $\vec r_{12} = \vec r - \vec r'$, reulting in $\md^3 \vec r_{12} = \md^3 r$, and allowing us to rewrite the integral as
\[ \hat H_b = \frac{e^2 N_e^2}{2\Omega^2} \iint \frac{e^{-\mu |\vec r_{12}|}}{|\vec r_{12}|} \dr_{12} \dr' = \frac{e^2 N_e^2}{2\Omega} \int \frac{e^{-\mu |\vec r_{12}|}}{|\vec r_{12}|} \dr_{12} . \]
%
Here we have used that $\int \! \dr = \Omega$. We change to spherical coordinates and the lack of angle 
dependencies yields a factor $4\pi$, resulting in
\[ \hat H_b = \frac{4\pi e^2 N_e^2}{2\Omega} \int_0^\infty re^{-\mu r} \, \md r . \]
%

}
 \end{small}
 }


\frame[containsverbatim]
{
  \frametitle{The electron gas, total energy}
\begin{small}
{\scriptsize
Solving by partial integration
\[ \int_0^\infty re^{-\mu r} \, \md r = \left[ -\frac{r}{\mu} e^{-\mu r} \right]_0^\infty + \frac{1}{\mu} \int_0^\infty e^{-\mu r} \, \md r
%
= \frac{1}{\mu} \left[ - \frac{1}{\mu} e^{-\mu r} \right]_0^\infty = \frac{1}{\mu^2}, \]
%
gives
\[
\hat H_b = \frac{e^2}{2} \frac{N_e^2}{\Omega} \frac{4\pi}{\mu^2} .
\]
%
The next term is 
\[ \hat H_{el-b} = -e^2 \sum_{i = 1}^N \int \frac{n(\vec r) e^{-\mu |\vec r - \vec x_i|}}{|\vec r - \vec x_i|} \dr . \]
%
Inserting  $n(\vec r)$ and changing variables in the same way as in the previous integral $\vec y = \vec r - \vec x_i$, we get $\md^3 \vec y = \md^3 \vec r$. This gives
\[ \hat H_{el-b} = -\frac{e^2 N_e}{\Omega} \sum_{i = 1^N} \int \frac{e^{-\mu |\vec y|}}{|\vec y|} \, \md^3 \vec y
%
=  -\frac{4\pi e^2 N_e}{\Omega} \sum_{i = 1}^N \int_0^\infty y e^{-\mu y} \md y. \]
%
The answer is 
\[ \hat H_{el-b} = -\frac{4\pi e^2 N_e}{\Omega} \sum_{i = 1}^N \frac{1}{\mu^2}, \]
which gives
\[
\hat H_{el-b} = -e^2 \frac{N_e^2}{\Omega} \frac{4\pi}{\mu^2} .
\]

}
 \end{small}
 }


\frame[containsverbatim]
{
  \frametitle{The electron gas, total energy}
\begin{small}
{\scriptsize
Finally, we need to evaluate $\hat H_{el}$. This term reads
\[ \hat H_{el} = \sum_{i=1}^{N_e} \frac{\hat{\vec p}_i^2}{2m_e} + \frac{e^2}{2} \sum_{i \neq j} \frac{e^{-\mu |\vec r_i - \vec r_j|}}{\vec r_i - \vec r_j} . \]
%
The last term represents the repulsion between two electrons. It is a central symmetric interaction
and is translationally invariant. The potential is given by the expression
\[ v(|\vec r|) = e^2 \frac{e^{\mu|\vec r|}}{|\vec r|},.\]
}
 \end{small}
 }


\frame[containsverbatim]
{
  \frametitle{The electron gas, total energy}
\begin{small}
{\scriptsize
The results becomes
\[ \int v(|\vec r|) e^{-i \vec q \cdot \vec r} \dr =
e^2 \int \frac{e^{\mu |\vec r|}}{|\vec r|} e^{-i \vec q \cdot \vec r} \dr =
e^2 \frac{4\pi}{\mu^2 + q^2} , \]
%
which gives us
\begin{align*}
\hat H_{el} &= \sum_{\vec k \sigma} \frac{\hbar^2 k^2}{2m} \hat a_{\vec k \sigma}^\dagger \hat a_{\vec k \sigma} +
\frac{e^2}{2\Omega} \sum_{\sigma \sigma'} \sum_{\vec k \vec p \vec q} \frac{4\pi}{\mu^2 + q^2}
\hat a_{\vec k + \vec q, \sigma}^\dagger \hat a_{\vec p - \vec q, \sigma'}^\dagger \hat a_{\vec p \sigma'} \hat a_{\vec k \sigma} \\
%
&= \sum_{\vec k \sigma} \frac{\hbar^2 k^2}{2m_e} \hat a_{\vec k \sigma}^\dagger \hat a_{\vec k \sigma} +
\frac{e^2}{2\Omega} \sum_{\sigma \sigma'} \sum_{\substack{\vec k \vec p \vec q \\ q \neq 0}} \frac{4\pi}{q^2}
\hat a_{\vec k + \vec q, \sigma}^\dagger \hat a_{\vec p - \vec q, \sigma'}^\dagger \hat a_{\vec p \sigma'} \hat a_{\vec k \sigma} + \\
&\quad\,\,
\frac{e^2}{2\Omega} \sum_{\sigma \sigma'} \sum_{\vec k \vec p} \frac{4\pi}{\mu^2}
\hat a_{\vec k, \sigma}^\dagger \hat a_{\vec p, \sigma'}^\dagger \hat a_{\vec p \sigma'} \hat a_{\vec k \sigma} ,
\end{align*}
%
where in the last sum we have split the sum over $\vec q$ in two parts, one with $\vec q\ne 0$ and one with 
$\vec q=0$. In the first term we also let $\mu\rightarrow 0$.
}
 \end{small}
 }


\frame[containsverbatim]
{
  \frametitle{The electron gas, total energy}
\begin{small}
{\scriptsize
The last term has the following set of creation and annihilation operatord
\[ \hat a_{\vec k, \sigma}^\dagger \hat a_{\vec p, \sigma'}^\dagger \hat a_{\vec p \sigma'} \hat a_{\vec k \sigma} =
- \hat a_{\vec k, \sigma}^\dagger \hat a_{\vec p, \sigma'}^\dagger \hat a_{\vec k \sigma} \hat a_{\vec p \sigma'} =
- \hat a_{\vec k, \sigma}^\dagger \hat a_{\vec p \sigma'} \delta_{\vec p \vec k} \delta_{\sigma \sigma'} + \hat a_{\vec k, \sigma}^\dagger \hat a_{\vec k \sigma} \hat a_{\vec p, \sigma'}^\dagger \hat a_{\vec p \sigma'} , \]
which gives
\[ \sum_{\sigma \sigma'} \sum_{\vec k \vec p} \hat a_{\vec k, \sigma}^\dagger \hat a_{\vec p, \sigma'}^\dagger \hat a_{\vec p \sigma'} \hat a_{\vec k \sigma} =
\hat N^2 - \hat N , \]
where we have used the expression for the number operator.  The term to the first power in $\hat N$ 
goes to zero in the thermodynamic limit since we are interested in the energy per electron $E_0/N_e$. This term will then be proportional with $1/(\Omega \mu^2)$. 
In the thermodynamical limit $\Omega\rightarrow \infty$ we can set this term equal to zero.
}
 \end{small}
 }


\frame[containsverbatim]
{
  \frametitle{The electron gas, total energy}
\begin{small}
{\scriptsize
We then get
%
\[ \hat H_{el} = \sum_{\vec k \sigma} \frac{\hbar^2 k^2}{2m} \hat a_{\vec k \sigma}^\dagger \hat a_{\vec k \sigma} +
\frac{e^2}{2\Omega} \sum_{\sigma \sigma'} \sum_{\substack{\vec k \vec p \vec q \\ q \neq 0}} \frac{4\pi}{q^2}
\hat a_{\vec k + \vec q, \sigma}^\dagger \hat a_{\vec p - \vec q, \sigma'}^\dagger \hat a_{\vec p \sigma'} \hat a_{\vec k \sigma} +
\frac{e^2}{2} \frac{N_e^2}{\Omega} \frac{4\pi}{\mu^2}. \]
%
The total Hamiltonian is $\hat H = \hat H_{el} + \hat H_{b} + \hat H_{el-b}$. 
Collecting all our terms we end up with
\[
\hat H_0 = \sum_{\vec k \sigma} \frac{\hbar^2 k^2}{2m_e} \hat a_{\vec k \sigma}^\dagger \hat a_{\vec k \sigma},
\]
and
\[
\hat H_I = \frac{e^2}{2\Omega} \sum_{\sigma \sigma'} \sum_{\substack{\vec k \vec p \vec q \\ q \neq 0}} \frac{4\pi}{q^2}
\hat a_{\vec k + \vec q, \sigma}^\dagger \hat a_{\vec p - \vec q, \sigma'}^\dagger \hat a_{\vec p \sigma'} \hat a_{\vec k \sigma},
\]
}
 \end{small}
 }


%



\end{document}




















